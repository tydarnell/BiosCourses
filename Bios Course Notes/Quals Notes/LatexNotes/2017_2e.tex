\documentclass{article}
\usepackage[utf8]{inputenc}
\usepackage[english]{babel}
\usepackage [autostyle, english = american]{csquotes}
\MakeOuterQuote{"}
\usepackage{graphicx}
\usepackage{enumerate}
\usepackage{float}
\graphicspath{ {} }
\usepackage{mathtools}
\usepackage{amsmath, amsthm, amssymb, amsfonts}
\usepackage{caption}
\usepackage{bm}
\usepackage{fancyhdr}
\pagestyle{fancy}
\fancyhf{}
\rhead{Ty Darnell}


% For derivatives
\newcommand{\deriv}[1]{\frac{\mathrm{d}}{\mathrm{d}x} (#1)}

% For partial derivatives
\newcommand{\pderiv}[2]{\frac{\partial #1}{\partial #2}}

% Integral dx
\newcommand{\dx}{\mathrm{d}x}
\newcommand{\cd}{\overset{d}{\to}}
\newcommand{\cp}{\overset{p}{\to}}
\newcommand{\B}{\beta}
\newcommand{\e}{\epsilon}
\newcommand{\limn}{\lim_{n\to \infty}}
\newcommand{\lm}{\lambda}
\newcommand{\sg}{\sigma}
\newcommand{\hb}{\hat{\beta}}
\newcommand{\sumn}{\sum_{i=1}^{n}}
\newcommand{\hth}{\hat{\theta}}
\newcommand{\lra}{\Leftrightarrow}
\newcommand{\prodn}{\prod_{i=1}^{n}}
\newcommand{\dll}[1]{\dfrac{\partial\ell}{\partial{#1}}}
\newcommand{\mle}{\hat{\theta}_{MLE}}
\newcommand{\mm}{\hat{\theta}_{MM}}
\newcommand{\sumx}{\sum_{i=1}^{n}x_i}
\newcommand{\ta}{\theta}
\newcommand{\qe}{ \ ?\ }
\newcommand{\dt}{\pderiv{}{\ta}}
\newcommand{\lt}[1]{\log(f(#1|\ta))}
\newcommand{\lx}{\lambda(x)}
\newcommand{\samp}{X_1,\dots,X_n \sim}
\newcommand{\te}{\theta_1}
\newcommand{\xm}{x_{(1)}}
\newcommand{\sn}{(\sg^2)}
\newcommand{\pow}{\B(\ta)}
\newcommand{\hyp}[2]{H_0: #1 \text{ vs } H_1: #2}
\newcommand{\pois}[2]{\dfrac{e^{-#1}{#1}^{#2}}{{#2}!}}
\newcommand{\mlr}{\dfrac{f(x|\ta_2)}{f(x|\ta_1)}}
\newcommand{\al}{\alpha}
\newcommand{\bx}{\bar{x}}
\allowdisplaybreaks
\begin{document}
\begin{flushleft}

	\section*{2017 2e}
	
	\item 
\begin{multline*}\\
S(t)=P(Y_1>T)=\exp\left(-t/\B \right) \text{ (from part b)}\\
\text{Let } \tau{\B}=\exp\left(-t/\B \right)=S(t)\\
\text{WTS: }  E(V_1|U) \text{ is an unbiased estimator of } S(t)\\
E(E(V_1|U))=E(V_1)\\
E(V_1)=S(t) \text{ from part c}\\
\text{Thus } E(E(V_1|U))=S(t)\\
\text{Therefore } E(V_1|U) \text{ is an unbiased estimator of } S(t)\\
\text{WTS } U=\sum Y_i \text{ is a css for } \B\\
f_y(y|\B)=\B^{-n}\exp\left(-\sum Y_i/\B \right) , y>0 \B>0\\
h(x)=I(y>0) \quad c(\B)=\B^{-n}\\
w(\B)=-1/\B \quad t(x)=\sum Y_i\\
\text{Thus } f(y|\B)=h(x)c(\B)\exp\left(w(\B)t(x)\right),  0<\B<\infty\\
T(x)=\sum Y_i=U \text{ is complete because:}\\
\{w(\B):\B\in (0,\infty)\} \text{ contains an open set in } R^1\\
w(\B)=-1/\B,\B>0=(-\infty,0) \text{ in } R^{-1} \text{ (the range contains an open set in } R^1)\\
\text{We have: } V_1 \text{ an unbiased estimator of } \tau(\B)=S(t), U \text{ a CSS for } \B\\
\text{Thus by lehmann-scheffe thm } E(V_1|U) \text{ is the UMVUE for } \tau(\B)=S(t)\\
\end{multline*}


\end{flushleft}
\end{document}
