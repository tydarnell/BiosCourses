\documentclass{article}
\usepackage[utf8]{inputenc}
\usepackage[english]{babel}
\usepackage{graphicx}
\usepackage{enumerate}
\usepackage{float}
\graphicspath{ {} }
\usepackage{mathtools}
\usepackage{amsmath, amsthm, amssymb, amsfonts}
\usepackage{caption}
\usepackage{fancyhdr}
\pagestyle{fancy}
\fancyhf{}
\rhead{Ty Darnell}
\lhead{663 Homework 1}

% For derivatives
\newcommand{\deriv}[1]{\frac{\mathrm{d}}{\mathrm{d}x} (#1)}

% For partial derivatives
\newcommand{\pderiv}[2]{\frac{\partial}{\partial #1} (#2)}

% Integral dx
\newcommand{\dx}{\mathrm{d}x}
\allowdisplaybreaks
\begin{document}
\begin{flushleft}

	\section*{Problem 1}
\begin{enumerate}[(a)]
	\item 
\begin{multline*}
\text{Reduced row echelon form of } \boldsymbol{A}=\begin{bmatrix}
1&0&2\\
0&1&-1\\
0&0&0\\
0&0&0
\end{bmatrix}\\
\text{Looking at the reduced row echelon form of A we can see columns 1 and 2 have pivots,}\\
\text{thus they are linearly independent from the rest.}\\
\text{The third column does not have a pivot so it is linearly dependent.}\\
\text{Reduced row echelon form of } \boldsymbol{B}=\begin{bmatrix}
1&0&0\\
0&1&0\\
0&0&1\\
0&0&0
\end{bmatrix}\\
\text{Since every column in the row reduced form of B has a pivot, the columns are linearly independent}\\
\end{multline*}
	\item 
\begin{multline*}
|\boldsymbol{A}-\lambda\boldsymbol{I}|=\bigg|\begin{bmatrix}
2-\lambda & 1\\
2 & 4-\lambda
\end{bmatrix}\bigg|=0\\
\lambda^2-\lambda+6=0\\
\text{eigenvalues } \lambda = 4.732 \text{ and } \lambda = 1.268\\
\lambda = 4.732 \text{ normalized eigenvector }=(-.3437,-.9391)^{'}\\
\lambda = 1.268 \text{ normalized eigenvector }=(-.8069,.5907)^{'}\\
\end{multline*}

\end{enumerate}

	\section*{Problem 2}
\begin{enumerate}[(a)]
	\item 
\begin{multline*}\\
\text{Let } Y=AX\\
A=\begin{bmatrix}
3&0&0\\
0&1&0\\
0&0&1
\end{bmatrix}\\
X\sim N(0,\Sigma)\\
Y\sim N(\mu^*,\Sigma^*)\\
\mu^*=A\mu=0\\
\Sigma^*=A\sigma A^{'}\\
=\begin{bmatrix}
3&1&1
\end{bmatrix}
\begin{bmatrix}
2&0&.6\\
0&2&.5\\
.6&.5&1
\end{bmatrix}
\begin{bmatrix}
3\\
1\\
1
\end{bmatrix}=25.6\\
Y\sim N(0,25.6)\\
\end{multline*}
	\item 
\begin{multline*}\\
\text{Partition } \mu \text{ as} \begin{bmatrix}
\mu_1\\
\mu_2\\
\end{bmatrix}\\
\mu =\begin{bmatrix}
0\\
0
\end{bmatrix}\\
\mu^*=\mu_1+\Sigma_{12}\Sigma_{22}^{-1}(x_3-\mu_2)\\
=0+\begin{bmatrix}
.6\\
.5
\end{bmatrix}
\begin{bmatrix}
1
\end{bmatrix}(3-0)\\
\mu^*=\begin{bmatrix}
1.8\\
1.5
\end{bmatrix}\\
\text{Partition } \Sigma \text{ as} \begin{bmatrix}
\Sigma_{11} & \Sigma_{12}\\
\Sigma_{21} & \Sigma_{22}
\end{bmatrix}\\
\Sigma=\left[\begin{array}{cc|c}
2 & 0 &.6 \\ 
0 & 2 &.5\\
\hline
.6 & .5&1
\end{array}\right]\\
\Sigma^*=\Sigma_{11}-\Sigma_{12}\Sigma_{22}^{-1}\Sigma_{21}\\
=\begin{bmatrix}
2&0\\
0&2
\end{bmatrix}-\begin{bmatrix}
.6\\
.5
\end{bmatrix}\begin{bmatrix}
1
\end{bmatrix}\begin{bmatrix}
.6&.5
\end{bmatrix}\\
\Sigma^*=\begin{bmatrix}
1.64 & -.3\\
-.3 & 1.75
\end{bmatrix}\\
(x_1,x_2|x_3=3)\sim N(\mu^*,\Sigma^*)\\
\end{multline*}

	\item 
\begin{multline*}\\
cov(ax+by,cw+dv)=ac \ cov(x,w)+ ad \ cov(x,v)+ bc \ cov(y,w)+ bd \ cov(y,v)\\
cov((x_1+2x_2),3x_2+x_3)\\
=3cov(x_1,x_2)+cov(x_1,x_3)+6cov(x_2,x_2)+2cov(x_2,x_3)\\
=3(0)+.6+6(2)+2(.5)\\
=13.6\\
\end{multline*}
	
\end{enumerate}

	\section*{Problem 3}
\begin{multline*}\\
\mu^*=E(Y)=E(a_1x_1+a_2x_2+\dots a_kx_k)\\
=a_1E(x_1)+\dots +a_kE(x_k)\\
\mu^*=\sum_{i=1}^{k}a_i\mu_i\\
\Sigma^*=Var(Y)=Var(a_1x_1+a_2x_2+\dots a_kx_k)\\
=a_1^2Var(x_1)+\dots +a_k^2Var(x_k)+2\left(\sum_{j>i}^{k}\sum_{i=1}^{k-1}a_ia_j(\Sigma_{i,j})\right)\\
\Sigma^*=\sum_{i=1}^{k}a_i^2\sigma_i^2+2\sum_{j>i}^{k}\sum_{i=1}^{k-1}a_ia_j(\Sigma_{i,j})\\
Y\sim N(\mu^*,\Sigma^*)\\
\end{multline*}

	\section*{Problem 4}
\begin{enumerate}[(a)]
	\item 
\begin{multline*}\\
y=X\beta+\epsilon\\
E(y)=X\beta\\
E(\hat{\beta_w})=E[(X^{'}V^{-1}X)^{-1}X^{'}V^{-1}y]\\
\text{Since } V \text{ is a matrix of constants and X is given we have:}\\ 
=(X^{'}V^{-1}X)^{-1}X^{'}V^{-1}E(y)\\
=(X^{'}V^{-1}X)^{-1}X^{'}V^{-1}X\beta\\
=X^{-1}(V^{-1})^{-1}X^{'-1}X^{'}V^{-1}X\beta\\
=X^{-1}VIV^{-1}X\beta\\
=X^{-1}IX\beta\\
=I\beta\\
E(\hat{\beta_w})=\beta\\
\end{multline*}
	\item 
\begin{multline*}\\
cov(y)=cov(X\beta+\epsilon)=cov(\epsilon)=\sigma^2V\\
cov(\hat{\beta}_w)=cov[(X^{'}V^{-1}X)^{-1}X^{'}V^{-1}y]\\
=[(X^{'}V^{-1}X)^{-1}X^{'}V^{-1}]cov(y)[(X^{'}V^{-1}X)^{-1}X^{'}V^{-1}]^{'}\\
=X^{-1}(V^{-1})^{-1}X^{'-1}X^{'}V^{-1}\sigma^2V[X^{-1}(V^{-1})^{-1}X^{'-1}X^{'}V^{-1}]^{'}\\
=\sigma^2X^{-1}VX^{'-1}X^{'}V^{-1}VV^{-1'}XX^{-1}V^{'}X^{-1'}\\
=\sigma^2X^{-1}VIIV^{-1'}IV^{'}X^{-1'}\\
=\sigma^2X^{-1}VV^{-1'}V^{'}X^{-1'}\\
=\sigma^2X^{-1}VX^{-1'}\\
cov(\hat{\beta}_w)=\sigma^2(X^{'}V^{-1}X)^{-1}\\
\end{multline*}
	\item 
\begin{multline*}\\
\text{Assuming normality of residuals:}\\
\hat{\beta_w}\sim N(\beta,\sigma^2(X^{'}V^{-1}X)^{-1})\\
\end{multline*}
	\item 
\begin{multline*}\\
\text{With this choice of V, you are dividing by the sample size}\\
\text{which gives you the variance of the distribution}\\
\end{multline*}

\end{enumerate}

\end{flushleft}
\end{document}
