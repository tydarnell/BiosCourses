\documentclass{article}
\usepackage[utf8]{inputenc}
\usepackage[english]{babel}
\usepackage{graphicx}
\usepackage{enumerate}
\usepackage{float}
\graphicspath{ {} }
\usepackage{mathtools}
\usepackage{amsmath, amsthm, amssymb, amsfonts}
\usepackage{caption}
\usepackage{fancyhdr}
\pagestyle{fancy}
\fancyhf{}
\rhead{Ty Darnell}
\lhead{663 Homework 3}

% For derivatives
\newcommand{\deriv}[1]{\frac{\mathrm{d}}{\mathrm{d}x} (#1)}

% For partial derivatives
\newcommand{\pderiv}[2]{\frac{\partial}{\partial #1} (#2)}

% Integral dx
\newcommand{\dx}{\mathrm{d}x}
\newcommand{\cd}{\overset{d}{\to}}
\newcommand{\cp}{\overset{p}{\to}}
\newcommand{\B}{\beta}
\newcommand{\e}{\epsilon}
\newcommand{\limn}{\lim_{n\to \infty}}
\newcommand{\lm}{\lambda}
\allowdisplaybreaks
\begin{document}
\begin{flushleft}

	\section*{Problem 1}
	
\begin{enumerate}[(a)]
	
	\item 	
\begin{tabular}{l|l|l|l|l|l}
	\hline
	Source&DF  &Sum of Squares  & Mean Square  & F Value & $Pr>F$  \\
	\hline
	Model&1  &2624.670184  &2624.670184  &137.906162& $\approx 0$ \\
	\hline
	Error& 96 & 1827.099916  & 19.0322908 &  \\
	\hline
	Corrected Total&97  &4451.7701  &  & \\
	\hline
\end{tabular}
\begin{multline*}\\
MST=\dfrac{SSH}{DFH} \quad MSE=\dfrac{SSE}{DFE} \quad F=\dfrac{MSH}{MSE}\\
Pr>F =1-pf(137.906162,1,96)=0 \quad \text{(using R)}\\
\end{multline*}

\item 
The model assumptions are:\\
Homogeneity of variance- every element of $\epsilon$ (error terms) has the same variance\\
Independence- each element of $\epsilon$ is independent of all others\\
Linearity- expected values of WGHT are linear function of the parameters. $E(y)=X\beta$\\
Existence - $\epsilon_i$ has finite first and second moments.\\
Gaussian errors- error terms are normally distributed. $\epsilon_i \sim N(0,\sigma_i^2)$\\

\item 
$H_0:\B_1=0$\\
Since the p-value is approximately 0, reject the null hypothesis and conclude that average daily exercise time is a significant predictor of weight loss. \\

\item 
The analysis suggests that neither variable is significant since both of the p-values are greater than $\alpha=.05$\\
This occurs in this added-last test because, after adjusting for running mileage, exercise time does not provide any additional useful information.\\ Running mileage and exercise time appear highly correlated.

\end{enumerate}
\pagebreak
	\section*{Problem 3 b}
	
\begin{enumerate}[(i)]
\item 
\item $H_0: Residuals are normally distributed$\\
$H_1: Residuals are not normally distributed$\\
$\alpha=.05$\\
p-value=$.242$\\
Since p-value$>\alpha$ fail to reject $H_0$ thus there is not sufficient evidence to conclude residuals are not normally distributed. Therefore we have met the Gaussian assumption

\item Based on the histogram, the studentized residuals appear normally distributed, centered around 0.

\item Based on the plot of the studentized residuals vs the predicted values, there appears to be homoscedasticity, and the mean of the residuals appears constant.


	
\end{enumerate}


\end{flushleft}
\end{document}
