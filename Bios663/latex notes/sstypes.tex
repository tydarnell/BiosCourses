\documentclass{article}
\usepackage[utf8]{inputenc}
\usepackage[english]{babel}
\usepackage [autostyle, english = american]{csquotes}
\MakeOuterQuote{"}
\usepackage{graphicx}
\usepackage{enumerate}
\usepackage{float}
\graphicspath{ {} }
\usepackage{mathtools}
\usepackage{amsmath, amsthm, amssymb, amsfonts}
\usepackage{caption}
\usepackage{bm}


% For derivatives
\newcommand{\deriv}[1]{\frac{\mathrm{d}}{\mathrm{d}x} (#1)}

% For partial derivatives
\newcommand{\pderiv}[2]{\frac{\partial}{\partial #1} (#2)}

% Integral dx
\newcommand{\dx}{\mathrm{d}x}
\newcommand{\cd}{\overset{d}{\to}}
\newcommand{\cp}{\overset{p}{\to}}
\newcommand{\B}{\beta}
\newcommand{\e}{\epsilon}
\newcommand{\limn}{\lim_{n\to \infty}}
\newcommand{\lm}{\lambda}
\newcommand{\sg}{\sigma}
\newcommand{\hb}{\hat{\beta}}
\newcommand{\sumn}{\sum_{i=1}^{n}}
\newcommand{\hth}{\hat{\theta}}
\newcommand{\lra}{\Leftrightarrow}
\newcommand{\prodn}{\prod_{i=1}^{n}}
\allowdisplaybreaks
\begin{document}
\begin{flushleft}
\section*{SS Types}
$SS(AB | A, B) = SS(A, B, AB) – SS(A, B)$\\
$SS(A | B, AB) = SS(A, B, AB) – SS(B, AB)$\\
$SS(B | A, AB) = SS(A, B, AB) – SS(A, AB)$\\
$SS(A | B) = SS(A, B) – SS(B)$\\
$SS(B | A) = SS(A, B) – SS(A)$\\

The notation shows the incremental differences in sums of squares\\  
$SS(AB | A, B)$ represents the sum of squares for interaction after the main effects\\ 
$SS(A | B)$ is the sum of squares for the A main effect after the B main effect and ignoring interactions\\

The different types of sums of squares then arise depending on the stage of model reduction at which they are carried out.\\


\section*{Type I}
\textbf{Type I}: $SS(A)$ for A, $SS(B | A)$ for B,
$SS(AB | B, A)$ for interaction AB\\
Tests  ME of A, followed by the ME of B after the ME of A, followed by AB after the MEs. Not great with unbalanced data\\
\section*{Type III}
\textbf{Type III}: $SS(A | B, AB)$ for A, $SS(B | A, AB)$ for B\\
Tests for ME after the other ME and interact. Good for signif interacts\\
not great for ME\\
\section*{Summary}
Data balanced, the factors orthogonal: all types same\\
Usually the hypothesis of interest is about the significance of one factor while controlling for the level of the other factors (Type II,III)\\
\section*{Type II}
Type II: $SS(A | B)$ for A, $SS(B | A)$ for B\\
Tests for each ME after the other ME.\\
no significant interaction.  test for interaction first $(SS(AB | A, B))$ and only if AB is not significant, continue with the analysis for main effects\\
Computationally, this is equivalent to running a type I analysis with different orders of the factors, and taking the appropriate output\\

\end{flushleft}
\end{document}
