\documentclass{article}
\usepackage[utf8]{inputenc}
\usepackage[english]{babel}
\usepackage{graphicx}
\usepackage{float}
\graphicspath{ {} }
\usepackage{mathtools}
\usepackage{amsmath, amsthm, amssymb, amsfonts}
\usepackage{caption}
\usepackage{fancyhdr}
\pagestyle{fancy}
\fancyhf{}
\rhead{Ty Darnell}
\lhead{Sup and Inf Notes}
\begin{document}
\begin{flushleft}
\section{Supremum}
\textbf{Bounded Above}: Let $S$ be a set of real numbers. If $\exists \ b$ such that $x\leq b \ \forall \ x \in S$, then $b$ is an \textbf{upper bound} for $S$. Then $S$ is \textbf{bounded above} by $b$. \medbreak
\textbf{Supremum}(or least upper bound): Let $S$ be a set of real numbers bounded above. $b$ is a least upper bound for $S$ ($b=sup \ S$) if:
\begin{itemize}
\item $b$ is an upper bound for $S$
\item No number less than $b$ is an upper bound for $S$
\end{itemize}
If $S$ has a \textbf{maximum element} then $max \ S= sup \ S$ \medbreak
\section{The Completeness Axiom}
\textbf{The Completeness Axiom}: Every nonempty set $S$ of real numbers which is bounded above has a sup; that is, there is a real number $b$ such that $b=sup \ S$.\medbreak
It follows that every nonempty set of real numbers which is \textbf{bounded below} has an \textbf{infimum}. \medbreak
\section{Some Properties of the Supremum}
\textbf{Approximation property}: Let $S$ be a nonempty set of real numbers with a supremum, $b= sup \ S$. Then $\forall \ a<b \ \exists \ x \in S$ such that $a<x\leq b$. \medbreak
\textbf{Additive Property}: Given nonempty subsets $A$ and $B$ of $R$, let $C$ denote the set:
\[C=\left\{x+y:x\in A, y \in B \right\}
\]
If each of $A$ and $B$ has a supremum, then C has a supremum and:
\[sup \ C= sup \ A + sup \ B
\]
\textbf{Comparison Property}: Given nonempty subsets $S$ and $T$ of $R$ such that $s \leq t \ \forall \ s \in S , t \in T$. If $T$ has a supremum then $S$ has a supremum and:
\[sup \ S \leq sup \ T
\]


\end{flushleft}
\end{document}
