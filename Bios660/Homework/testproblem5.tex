\documentclass{article}
\usepackage[utf8]{inputenc}
\usepackage[english]{babel}
\usepackage{graphicx}
\usepackage{enumerate}
\usepackage{float}
\graphicspath{ {} }
\usepackage{mathtools}
\usepackage{amsmath, amsthm, amssymb, amsfonts}
\usepackage{caption}
\usepackage{fancyhdr}
\pagestyle{fancy}
\fancyhf{}
\rhead{Ty Darnell}

% For derivatives
\newcommand{\deriv}[1]{\frac{\mathrm{d}}{\mathrm{d}x} (#1)}

% For partial derivatives
\newcommand{\pderiv}[2]{\frac{\partial}{\partial #1} (#2)}

% Integral dx
\newcommand{\dx}{\mathrm{d}x}
\begin{document}
\begin{flushleft}
\section{Problem 5}
Suppose $P(A \cup B) = P(A \cap B)$\\
We can write $P(A)=P(A\cap B)+P(A \cap B^c)$\\
and $P(B)=P(A \cap B)+P(A^c \cap B)$\\
Then \[P(A)+P(B)=2P(A\cap B)+P(A \cap B^c)+P(A^c\cap B) \label{1} \tag{1}\]\\
By definition $P(A \cup B) = P(A)+P(B)-P(A \cap B)$\\
Since $P(A \cup B)=P(A \cap B)$ we have:\\
\[P(A)+P(B)=2P(A \cap B) \label{2} \tag{2}\]\\
Combining \eqref{1} and \eqref{2} we have:\\
$2P(A \cap B)=2P(A \cap B)+P(A\cap B^c)+P(A^c \cap B)$\\
Thus $P(A \cap B^c)+P(A^c\cap B)=0$\\
Since probability is nonnegative\\
$P(A\cap B^c)=0$ and\\
$P(A^c \cap B)=0$\\
Thus $P(A)=P(B)$




\end{flushleft}
\end{document}
