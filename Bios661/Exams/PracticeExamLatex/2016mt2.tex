\documentclass{article}
\usepackage[utf8]{inputenc}
\usepackage[english]{babel}
\usepackage [autostyle, english = american]{csquotes}
\MakeOuterQuote{"}
\usepackage{graphicx}
\usepackage{enumerate}
\usepackage{float}
\graphicspath{ {} }
\usepackage{mathtools}
\usepackage{amsmath, amsthm, amssymb, amsfonts}
\usepackage{caption}
\usepackage{bm}
\usepackage{fancyhdr}
\pagestyle{fancy}
\fancyhf{}
\rhead{Ty Darnell}
\lhead{2016 Midterm 2}

% For derivatives
\newcommand{\deriv}[1]{\frac{\mathrm{d}}{\mathrm{d}x} (#1)}

% For partial derivatives
\newcommand{\pderiv}[2]{\frac{\partial #1}{\partial #2}}

% Integral dx
\newcommand{\dx}{\mathrm{d}x}
\newcommand{\cd}{\overset{d}{\to}}
\newcommand{\cp}{\overset{p}{\to}}
\newcommand{\B}{\beta}
\newcommand{\e}{\epsilon}
\newcommand{\limn}{\lim_{n\to \infty}}
\newcommand{\lm}{\lambda}
\newcommand{\sg}{\sigma}
\newcommand{\hb}{\hat{\beta}}
\newcommand{\sumn}{\sum_{i=1}^{n}}
\newcommand{\hth}{\hat{\theta}}
\newcommand{\lra}{\Leftrightarrow}
\newcommand{\prodn}{\prod_{i=1}^{n}}
\newcommand{\dll}[1]{\dfrac{\partial\ell}{\partial{#1}}}
\newcommand{\mle}{\hat{\theta}_{MLE}}
\newcommand{\mm}{\hat{\theta}_{MM}}
\newcommand{\sumx}{\sum_{i=1}^{n}x_i}
\newcommand{\ta}{\theta}
\newcommand{\qe}{ \ ?\ }
\newcommand{\dt}{\pderiv{}{\ta}}
\newcommand{\lt}[1]{\log(f(#1|\ta))}
\allowdisplaybreaks
\begin{document}
\begin{flushleft}

	\section*{Problem 1}
	
\begin{enumerate}[(a)]
	
	\item 
\begin{multline*}\\
(X_1,X_2,X_3)\sim Multinomial(n,p=[(1-\theta)^2,2\theta(1-\theta),\ta^2])\\
\ell(\ta|x)=\log\left(\dfrac{n!}{x_1!x_2!x_3!}[(1-\theta)^2]^{x_1}2\theta(1-\theta)^{x_2}(\ta^2)^{x_3}\right)\\
=\log\left(\dfrac{n!}{x_1!x_2!x_3!}\right)+2x_1\log(1-\theta)+x_2\log(2\theta(1-\theta))+2x_3\log(\ta)\\
=\log\left(\dfrac{n!}{x_1!x_2!x_3!}\right)+ (2x_1+x_2)\log(1-\theta)+(2x_3+x_2)\log(\theta)+x_2\log(2)\\
\dll{\theta}=-\dfrac{2x_1+x_2}{1-\ta}+\dfrac{2x_3+x_2}{\ta}=0\\
\dfrac{2x_1+x_2}{1-\ta}=\dfrac{2x_3+x_2}{\ta}\\
\dfrac{1-\ta}{\ta}=\dfrac{2x_1+x_2}{2x_3+x_2}\\
\dfrac{1}{\theta}=\dfrac{2x_1+x_2}{2x_3+x_2}+1\\
\ta=\dfrac{2x_3+x_2}{2x_3+2x_2+2x_1}\\
\text{Since } x_1+x_2+x_3=n:\\
\hth=\dfrac{2x_3+x_2}{2n}\\
\dll{\ta^2}^2=-\dfrac{2x_1+x_2}{(1-\ta)^2}-\dfrac{2x_3+x_2}{\ta^2}\\
\text{Plugging in } \hth:\\
-\left(\dfrac{2x_3+x_2}{(1-(2x_1+x_2)/2n)^2}+\dfrac{2x_3+x_2}{((2x_3+x_2)/2n)^2}\right)\\
-\left(\dfrac{2x_3+x_2}{((2n-2x_1+x_2)/2n)^2}+\dfrac{2x_3+x_2}{((2x_3+x_2)/2n)^2}\right)<0\\
\text{Thus } \hth \text{ is the MLE}\\
E(\hth)=E\left(\dfrac{2x_3+x_2}{2n} \right)=\dfrac{1}{2n}[2E(X_3)+E(X_2)]\\
=\dfrac{2n\ta^2+2n\ta(1-\ta)}{2n}=\ta^2+\ta(1-\ta)=\ta^2+\ta-\ta^2=\ta\\
E(\hth)=\theta, \text{ Thus } \hth \text{ is unbiased}\\
\end{multline*}

	\item 
\begin{multline*}\\
\dll{\theta}=-\dfrac{2x_1+x_2}{1-\ta}+\dfrac{2x_3+x_2}{\ta}\\
\dll{\ta^2}^2=-\left(\dfrac{2x_1+x_2}{(1-\ta)^2}+\dfrac{2x_3+x_2}{\ta^2}\right)\\
-E[(\pderiv{^2}{\theta^2}\log f(x|\theta))]=-E\left(-\dfrac{2x_1+x_2}{(1-\ta)^2}-\dfrac{2x_3+x_2}{\ta^2}\right)\\
=\left[\dfrac{1}{(1-\ta)^2}2(E(X_1)+E(X_2))+\dfrac{1}{\ta^2}(2E(X_3)+E(X_2))\right]\\
=\left[\dfrac{1}{(1-\ta)^2}(2n(1-\theta^2)+2n\ta(1-\ta))+\dfrac{1}{\ta^2}(2n\ta^2+2n\ta(1-\ta))\right]\\
=2n\left(2+\dfrac{\ta}{1-\ta}+\dfrac{1-\ta}{\ta}\right)=2n\left(\dfrac{\ta}{1-\ta}+\dfrac{1}{\ta}+1\right)\\
=2n\left(\dfrac{\ta^2+(1-\ta)+\ta(1-\ta)}{\ta(1-\ta)}\right)\\
-E[(\pderiv{^2}{\theta^2}\log f(x|\theta))]=\dfrac{2n}{\ta(1-\ta)}\\
CRLB=\dfrac{1}{-E[(\pderiv{^2}{\theta^2}\log f(x|\theta))]}=\dfrac{\ta(1-\ta)}{2n}\\
\end{multline*}

\pagebreak
	\item 
\begin{multline*}\\
\text{Show } \sqrt{n}(X_3/n-\ta^2)\cd N(0,\ta^2(1-\ta^2))\\
\text{Since the outcome is either AA,Aa, or aa:}\\
P(AA)+P(Aa)+P(aa)=1\\
P(aa)=1-[P(AA)+P(Aa)]=\theta^2\\
\text{Let } A=aa\\
A\sim Bern(\theta^2)\\
X_3=\sumn a_i\\
X_3 \text{ can be written as a sum of n iid } Bern(\theta^2) \text{ rvs each with:}\\
\mu=\theta^2 \quad \sigma^2=\ta^2(1-\ta^2)\\
\text{Let } Z_n=\sqrt{n}(X_3/n-\ta^2)\\
\text{By CLT: }Z_n\cd N(0,\ta^2(1-\ta^2))\\
\text{Find } \sigma^2 \text{ such that } \sqrt{n}(\sqrt{X_3/n}-\ta)\cd N(0,\sg^2)\\
\text{Using delta method:}\\
\sqrt{n}\{g(X_3/n)-g(\ta^2)\}\cd N(0,\{g^{'}(\ta^2)\}^2\ta^2(1-\ta^2))\\
g(x)=x^{1/2} \quad g^{'}(x)=(1/2)x^{-1/2}\\
\sqrt{n}\{\sqrt{X_3/n}-\ta^2\}\cd N(0,\{(1/2)(\ta^{2})^{-1/2}\}^2\ta^2(1-\ta^2))=N(0,(1-\ta^2)/4)\\
\sigma^2/n=(1-\ta^2)/4n\\
\dfrac{1-\ta^2}{4n} \qe \dfrac{\ta(1-\ta)}{2n}\\
1-\ta^2 \qe 2\ta(1-\ta)\\
1-\theta^2=P(AA)+P(Aa)\\
2\ta(1-\ta)=P(Aa)\\
\text{Since } P(AA+P(Aa)>P(Aa)\\
1-\ta^2 > 2\ta(1-\ta)\\
\text{Thus } \dfrac{\sigma^2}{n}>\dfrac{\ta(1-\ta)}{2n}\\
\end{multline*}

\end{enumerate}
\pagebreak
	\section*{Problem 2}
\begin{enumerate}[(a)]
	
	\item 
\begin{multline*}\\
T_1,T_2 \text{ are SS, } U \text{ unbiased estimator of } \ta \quad V_1=E(U|T_1)\\
\text{Two ways to solve:}\\
\textbf{1) } \text{Let } \tau(\ta)=\ta\\
U \text{ is an unbiased estimator of } \tau(\ta) \text{ and } T_1 \text{ is an SS for } \ta\\
\text{Let } \phi(T_1)=E(U|T_1)\\
\text{Then } E(\phi(T_1))=\tau{\ta}=\ta\\
\text{Thus }V_1 \text{ is an unbiased estimator of } \tau(\ta) \text{ by Rao-Blackwell Thm}\\
\textbf{2) } E(V_1)=E(E(U|T_1))=E(U)=\theta\\
\text{Thus } V_1 \text{ is an unbiased estimator of } \ta\\
V_2=E(V_1|T_2)\\
E(V_2)=E(E(V_1|T_2))=E(V_1)=\ta\\
\text{Thus } V_2 \text{ is an unbiased estimator of } \ta\\
\end{multline*}

	\item 
\begin{multline*}\\
\text{WTS: } Var(V_2)\leq Var(V_1)\\
\text{Let } \tau(\ta)=\ta \text{ and } \phi(T_2)=E(V_1|T_2)\\
V_2=E(V_1|T_2)=\tau(\ta)\\
\text{By Rao-Blackwell: } Var(\phi(T_2))\leq Var(V_1)\\
\text{Thus } Var(V_2)\leq Var(V_1)\\
\end{multline*}
	
\end{enumerate}
\pagebreak
	\section*{Problem 3}
	
\begin{enumerate}[(a)]
	\item 
\begin{multline*}\\
f(x|\B)=\dfrac{1}{\B}e^{-x/\B} \quad x>0, \ \B>0\\
\ell(\B|x)=-n\log(\B)-\B^{-1}\sumx\\
\dll{\B}=\dfrac{-n}{\B}+\dfrac{1}{\B^2}\sumx=0\\
\dfrac{n}{\B}=\dfrac{1}{\B^2}\sumx\\
\hat{\B}=\bar{X}\\
\dll{\B^2}^2=\dfrac{n}{\B^2}-\dfrac{2}{\B^3}\sumx\\
\text{Plugging in } \hat{\B}:\\
\dfrac{n}{\bar{X}^2}-\dfrac{2}{\bar{X}^3}n\bar{X}\\
=\dfrac{n-2n}{\bar{X^2}}\\
-\dfrac{n}{\bar{X}^2}<0\\
\text{Thus } \hat{\B} \text{ is the MLE}\\
\tau(\B)=\B^2=\ta\\
\text{By the invariance property, the MLE of } \ta \text{ is } \bar{X}^2\\
L(\ta|x)=\prodn\ta^{-1/2}\exp(-x_i(\ta)^{-1/2})\\
=\ta^{-n/2}\exp\left(-\ta^{-1/2}\sumx\right)\\
\ell(\ta|x)=(-n/2)\log(\ta)-\ta^{-1/2}\sumx\\
\dll{\ta}=\dfrac{-n}{2\ta}+\dfrac{1}{2\ta^{3/2}}\sumx=0\\
\dfrac{n}{2\ta}=\dfrac{1}{2\ta^{3/2}}\sumx\\
\dfrac{\ta^{3/2}}{\ta}=\dfrac{1}{n}\sumx\\
\ta^{1/2}=\dfrac{1}{n}\sumx\\
\hth=\dfrac{1}{n^2}\left(\sumx\right)^2\\
\hth=\bar{X}^2\\
\pderiv{^2}{\ta^2}=\dfrac{n}{2\ta^2}-\dfrac{3}{4\ta^{5/2}}\sumx\\
\end{multline*}

	\item 
\begin{multline*}\\
E(\bar{X}^2)=Var(\bar{X})+E(\bar{X})^2\\
=\B^2/n+\B^2=\dfrac{\B^2(n+1)}{n}\\
\hth^*=\dfrac{n}{n+1}\bar{X}^2\\
\text{Since } \bar{X} \text{ is a CSS (from exponential family)} \\
\text{ and } \hth^*=\dfrac{n}{n+1}\bar{X}^2 \text{ is an unbiased estimator for } \tau(\B)=\B^2 \text{ and:}\\
E\left(\dfrac{n}{n+1}\bar{X}^2|\bar{X}\right)=\dfrac{n}{n+1}\bar{X}^2\\
\text{By Lehmann-Sheffe Thm, } \hth^* \text{ is the UMVUE}\\
\end{multline*}
\pagebreak
	\item 
\begin{multline*}\\
Var(\hth^*)=Var\left(\dfrac{n}{n+1}\bar{X}^2\right) \quad \text{Let } W=\sumx \quad W\sim Gamma(n,\B)\\
=Var\left(\dfrac{n}{n+1}\dfrac{1}{n^2}W^2\right)
=\left(\dfrac{1}{n(n+1)}\right)^2Var(W^2)\\
=\dfrac{1}{n^2(n+1)^2}(E(W^4)-E(W^2)^2)\\
E(W^4)=\dfrac{\Gamma(n+4)}{\Gamma(n)}\B^4\quad E(W^2)=\dfrac{\Gamma(n+2)}{\Gamma(n)}\B^2\\
Var(\hth^*)=\dfrac{1}{n^2(n+1)^2}\left[\dfrac{\Gamma(n+4)}{\Gamma(n)}\B^4-\left(\dfrac{\Gamma(n+2)}{\Gamma(n)}\B^2\right)^2 \right]\\
=\dfrac{1}{n^2(n+1)^2}\B^4\left[\dfrac{\Gamma(n+4)}{\Gamma(n)}-\dfrac{\Gamma(n+2)^2}{\Gamma(n)^2} \right]\\
=\dfrac{1}{n^2(n+1)^2}\B^4\left(\dfrac{(n+3)(n+2)(n+1)n\Gamma(n)}{\Gamma(n)}-\dfrac{(n+1)^2(n^2)\Gamma(n)^2}{\Gamma(n)^2}\right)\\
=\B^4\dfrac{(n+3)(n+2)(n+1)n-(n+1)^2(n^2)}{n^2(n+1)^2}\\
=\B^4\left(\dfrac{(n+3)(n+2)}{n(n+1)}-1\right)\\
Var(\hth^*)=\B^4\left(\dfrac{4n+6}{n(n+1)}\right)\\
\dll{\B^2}^2=\dfrac{n}{\B^2}-\dfrac{2}{\B^3}\sumx\\
-E[\pderiv{^2}{\ta^2}\lt{X_1}]=-E\left(\dfrac{n}{\B^2}-\dfrac{2}{\B^3}\sumx\right)=-\dfrac{n}{\B^2}+\dfrac{2}{\B^3}E\left(\sumx\right)\\
=-\dfrac{n}{\B^2}+\dfrac{2}{\B^3}n\B=\dfrac{-n+2n}{\B^2}=\dfrac{n}{\B^2}\\
\text{Since } \ta=\B^2 \text{ we have: }\dfrac{n}{\B^4}\\
CRLB=1/\dfrac{n}{\B^4}=\dfrac{\B^4}{n}\\
Var(\hth^*)=\B^4\left(\dfrac{4n+6}{n(n+1)}\right)=\left(\dfrac{6}{n}-\dfrac{2}{n+1}\right)\B^4\approx \dfrac{4\B^4}{n}\geq \dfrac{\B^4}{n}\\
\text{alternatively as } n\to\infty: Var(\hth*)\approx \dfrac{4\B^4}{n}\quad \dfrac{4\B^4}{n}\geq \dfrac{\B^4}{n}\\
\text{Thus } Var(\hth^*) \text{ never reaches the CRLB}\\
\end{multline*}

	\item 
\begin{multline*}\\
L(\B|x)=\B^{-n}\exp\left(\B^{-1}\sumx\right)\\
\lambda(x)=\dfrac{L(\theta_0|x)}{L(\bar{x}|x)}\\
=\dfrac{\B_0^{-n}\exp\left(-\sumx/\B_0\right)}{\bar{x}^{-n}\exp\left(-\sumx/\bar{x} \right)}\\
=\left(\dfrac{\B_0}{\bar{x}}\right)^{-n}\dfrac{\exp\left(-\sumx/\B_0 \right)}{\exp\left(-n^{-1}\right)}\\
\lambda(x)=\left(\dfrac{\B_0}{\bar{x}}\right)^{-n}\exp\left(-\sumx/\B_0+n\right)\\
\end{multline*}

	\item 
\begin{multline*}\\
R=\{x:\lambda(x)\leq c\}\\
R^*=\{x:\bar{x}\leq c_1^* \text{ or } \bar{x}\geq c_2^* \}\\
\text{WTS: } R \text{ is equivalent to } R^*\\
\lambda(x)=\left(\dfrac{\B_0}{\bar{x}}\right)^{-n}\exp\left(-\sumx/\B_0+n\right)\\
=\left(\dfrac{\bar{x}}{\B_0}\right)^{n}\exp\left(-n\bar{x}/\B_0+n\right)\\
\text{Let } y=\dfrac{\bar{x}}{\B_0}\\
\lambda(x)=y^{n}\exp(-ny+n)\\
R=\{x:y^{n}\exp(-ny+n)\leq c\}\\
R=\{x:\log(y)-y\leq \log(c)/n-1\}\\
\log(y)-y \text{ is a concave function of y, thus: }\\
R \text{ is equivalent to } R^*\\
\end{multline*}

\end{enumerate}

\end{flushleft}
\end{document}
