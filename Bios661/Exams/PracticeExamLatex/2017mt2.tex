\documentclass{article}
\usepackage[utf8]{inputenc}
\usepackage[english]{babel}
\usepackage [autostyle, english = american]{csquotes}
\MakeOuterQuote{"}
\usepackage{graphicx}
\usepackage{enumerate}
\usepackage{float}
\graphicspath{ {} }
\usepackage{mathtools}
\usepackage{amsmath, amsthm, amssymb, amsfonts}
\usepackage{caption}
\usepackage{bm}
\usepackage{fancyhdr}
\pagestyle{fancy}
\fancyhf{}
\rhead{Ty Darnell}
\lhead{2017 Midterm 2}

% For derivatives
\newcommand{\deriv}[1]{\frac{\mathrm{d}}{\mathrm{d}x} (#1)}

% For partial derivatives
\newcommand{\pderiv}[2]{\frac{\partial #1}{\partial #2}}

% Integral dx
\newcommand{\dx}{\mathrm{d}x}
\newcommand{\cd}{\overset{d}{\to}}
\newcommand{\cp}{\overset{p}{\to}}
\newcommand{\B}{\beta}
\newcommand{\e}{\epsilon}
\newcommand{\limn}{\lim_{n\to \infty}}
\newcommand{\lm}{\lambda}
\newcommand{\sg}{\sigma}
\newcommand{\hb}{\hat{\beta}}
\newcommand{\sumn}{\sum_{i=1}^{n}}
\newcommand{\hth}{\hat{\theta}}
\newcommand{\lra}{\Leftrightarrow}
\newcommand{\prodn}{\prod_{i=1}^{n}}
\newcommand{\dll}[1]{\dfrac{\partial\ell}{\partial{#1}}}
\newcommand{\mle}{\hat{\theta}_{MLE}}
\newcommand{\mm}{\hat{\theta}_{MM}}
\newcommand{\sumx}{\sum_{i=1}^{n}x_i}
\newcommand{\ta}{\theta}
\newcommand{\qe}{ \ ?\ }
\newcommand{\dt}{\pderiv{}{\ta}}
\newcommand{\lt}[1]{\log(f(#1|\ta))}
\newcommand{\lx}{\lambda(x)}
\allowdisplaybreaks
\begin{document}
\begin{flushleft}

	\section*{Problem 1}
	
\begin{enumerate}[(a)]
	
	\item 
\begin{multline*}\\
X_1,\dots, X_{n_1} \sim N(\mu_1,\sigma^2)\\
E(c\sum_{i=1}^{n_1-1}(X_{i+1}-X_i)^2)=\sg^2\\
(X_{i+1}-X_i)\sim N(\mu_1-\mu_1,2\sigma^2)\\
\text{Let } Z=\dfrac{(X_{i+1}-X_i)}{\sqrt{2}\sigma}\\
Z\sim N(0,1) \quad Z^2\sim \chi^2_{1}\\
\sum_{i=1}^{n_1-1}Z^2\sim \chi^2_{n_1-1}\\
E(\sum Z^2)=n_1-1\\
E(\sum_{i=1}^{n_1-1}(X_{i+1}-X_i)^2)=(n_1-1)2\sg^2\\
\dfrac{1}{2(n_1-1)}E(\sum_{i=1}^{n_1-1}(X_{i+1}-X_i)^2)=\sg^2\\
c=\dfrac{1}{2(n_1-1)}\\
\end{multline*}

	\item 
\begin{multline*}\\
Y\sim N(\mu_2,\sg^2) \quad X\perp Y\\
S_1^2=\dfrac{1}{n_1-1}\sumn(X_i-\bar{X})^2\quad S_2^2=\dfrac{1}{n_2-1}\sumn(Y_i-\bar{Y})^2\\
S_p^2=aS_1^2+(1-a)S_2^2\\
\text{WTS: } E(S_p^2)=\sg^2\\
E(aS_1^2+(1-a)S_2^2)\\
=\left(\dfrac{a}{n_1-1}\right)(n_1-1)\sg^2+\left(\dfrac{1-a}{n_2-1}\right)(n_2-1)\sg^2\\
=\sg^2(a+1-a)=\sg^2\\
\end{multline*}

	\item 
\begin{multline*}\\
\text{Since } (n-1)S^2/\sg^2 \sim \chi^2_{n-1}\\
Var(\dfrac{n_1-1}{\sg^2}S_1^2)=2(n_1-1)\\
Var(\dfrac{n_2-1}{\sg^2}S_2^2)=2(n_2-1)\\
Var(S_1^2)=\dfrac{2(n_1-1)}{(n_1-1)^2}\sg^4=\dfrac{2\sg^2}{n_1-1}\\
Var(S_2^2)=\dfrac{2\sg^2}{n_2-1}\\
Var(S_p^2)=Var(aS_1^2+(1-a)S_2^2)\\
=a^2Var(S_1^2)+(1-a)^2Var(S_2^2)+2a(1-a)Cov(S_1^2,S_2^2)\\
=a^2\left(\dfrac{2\sg^2}{n_1-1}\right)+(1-a)^2\left(\dfrac{2\sg^2}{n_2-1}\right)+0\\
=\sg^4\left[2\left(\dfrac{a^2}{n_1-1}+\dfrac{(1-a)^2}{n_2-1}\right)\right]\\
Var(S_p^2)=\sigma^4g(a)\\
g(a)=2\left(\dfrac{a^2}{n_1-1}+\dfrac{(1-a)^2}{n_2-1}\right)\\
g^{\prime}(a)=2\left(\dfrac{2a}{n_1-1}-\dfrac{2(1-a)}{n_2-1}\right)=0\\
4\left(\dfrac{a}{n_1-1}-\dfrac{(1-a)}{n_2-1}\right)=0\\
\dfrac{a}{n_1-1}=\dfrac{1-a}{n_2-1}\\
\dfrac{1-a}{a}=\dfrac{n_2-1}{n_1-1}\\
a=\dfrac{n_1-1}{n_1+n_2-2}\\
g^{''}(a)=4\left(\dfrac{1}{n_1-1}+\dfrac{1}{n_2-1}\right)>0\\
\text{Thus } a=\dfrac{n_1-1}{n_1+n_2-2} \\
\text{Let } P=\sumn(X_i-\bar{X})^2 \quad Q=\sumn(Y_i-\bar{Y})^2\\
S_p^2=\dfrac{\dfrac{n_1-1}{n_1+n_2-2}}{n_1-1}P+\dfrac{\dfrac{n_2-1}{n_1+n_2-2}}{n_2-1}Q
=\dfrac{1}{n_1+n_2-2}(P+Q)\\
\end{multline*}	
\begin{multline*}\\
\text{Alternatively:}\\
S_p^2=a\dfrac{1}{n_1-1}\sumn(X_i-\bar{X})^2+(1-a)\dfrac{1}{n_2-1}\sumn(Y_i-\bar{Y})^2\\\text{Let } P=\sumn(X_i-\bar{X})^2 \quad Q=\sumn(Y_i-\bar{Y})^2\\
S_p^2=\dfrac{a}{n_1-1}P+\dfrac{1-a}{n_2-1}Q=\dfrac{1}{n_1+n_2-2}(P+Q)\\
\text{We have the following two equations which are equal to each other:}\\
1) \quad \dfrac{a}{n_1-1}=\dfrac{1}{n_1+n_2-2}\\
2) \quad \dfrac{1-a}{n_2-1}=\dfrac{1}{n_1+n_2-2} \\
\text{From } 1) \quad a=\dfrac{n_1-1}{n_1+n_2-2}\\
\text{From } 2) \quad 1-a=\dfrac{n_2-1}{n_1+n_2-2}\\
S_p^2=\dfrac{\dfrac{n_1-1}{n_1+n_2-2}}{n_1-1}P+\dfrac{\dfrac{n_2-1}{n_1+n_2-2}}{n_2-1}Q
=\dfrac{1}{n_1+n_2-2}(P+Q)\\
\end{multline*}

\end{enumerate}
\pagebreak5
	\section*{Problem 2}
\begin{enumerate}[(a)]
	
	\item 
\begin{multline*}\\
\tau(\ta)=e^{-\ta}\\
L(\ta|x)=\prodn \left(\dfrac{1}{x_i!} \right) \ta^{\sumx}e^{-n\ta}\\
\propto \ell(\ta|x)=\sumx \log(\ta)-n\ta\\
\dll{\ta}=\dfrac{\sumx}{\ta}-n=0\\
\mle=\bar{x}\\
\dll{\ta^2}^2=\dfrac{-\sumx}{\ta^2}<0\\
\text{Thus } \hth \text{ is the MLE}\\
\text{By invariance property:}\\
\hat{\tau(\ta)}_{MLE}=e^{-\mle}=e^{-\bar{x}}\\
\end{multline*}

	\item 
\begin{multline*}\\
\end{multline*}

	\item 
\begin{multline*}\\
\end{multline*}

	\item 
\begin{multline*}\\
\end{multline*}
	
\end{enumerate}


\end{flushleft}
\end{document}
