\documentclass{article}
\usepackage[utf8]{inputenc}
\usepackage[english]{babel}
\usepackage{graphicx}
\usepackage{float}
\graphicspath{ {} }
\usepackage{mathtools}
\usepackage{amsmath, amsthm, amssymb, amsfonts}
\usepackage{caption}
\usepackage{fancyhdr}
\pagestyle{fancy}
\fancyhf{}
\rhead{Ty Darnell}
\lhead{Induction Notes}

% For derivatives
\newcommand{\deriv}[1]{\frac{\mathrm{d}}{\mathrm{d}x} (#1)}

% For partial derivatives
\newcommand{\pderiv}[2]{\frac{\partial}{\partial #1} (#2)}

% Integral dx
\newcommand{\dx}{\mathrm{d}x}
\begin{document}
\begin{flushleft}
\section{PMI}
If $T$ is a subset of $\mathbb{N}$ such that:
\begin{enumerate}
\item $1\in T$
\item $\forall \ k \in \mathbb{N}$, if $k\in T$, then
$(k+1)\in T$, \\
Then $T=\mathbb{N}$
\end{enumerate}
Use induction to prove statements in the form of 
$(\forall \ n \in \mathbb{N})(P(n))$\medbreak
Goal is to prove $T = \mathbb{N}$ \medbreak
\begin{itemize}
\item Basis Step: Prove $P(1)$
\item Inductive Step: Prove $\forall \ k\in \mathbb{N}$ if
$P(k)$ is true, then $P(k+1)$ is true.
\item Conclude $P(n)$ is true $\forall \ n \in \mathbb{N}$
\end{itemize}
Start inductive step by assuming $P(k)$ is true.\\
This is called the \textbf{inductive hypothesis}\medbreak
The key is to discover how $P(k+1)$ is related to $P(k)$ for an arbitrary $k\in \mathbb{N}$
\section{Induction Proof Example}

\begin{align*}
\text{Proof by induction:}&\\
\forall \ n \in \mathbb{N} &\text{ let } P(n) \text{ be:}\\
\sum_{j=1}^{n}j&=\dfrac{n(n+1)}{2}\\
\textbf{Basis Step: } 1&=\dfrac{1(1+1)}{2}\\
&= 1\\
\text{Thus } &P(1) \text{ is true}\\
\textbf{Inductive Step: } \text{Let } k&\in \mathbb{N} \text{ and assume }
P(k) \text{is true:}\\
\sum_{j=1}^{k}j&=\dfrac{k(k+1)}{2} \label{1}\tag{1}\\
\text{We will prove } &P(k+1) \text{ is true:}\\
\sum_{j=1}^{k+1}j&=\dfrac{(k+1)((k+1)+1)}{2}\\
&=\dfrac{(k+1)(k+2)}{2}\label{2}\tag{2}\\
\text{add } &(k+1) \text{ to both sides of } \eqref{1}\\
\sum_{j=1}^{k}j +(k+1)&=\dfrac{k(k+1)}{2}+(k+1)\\
&=\dfrac{(k+1)(k+2)}{2}\\
&\text{Which is the same as } \eqref{2}\\
&\text{Hence the inductive step has been established}\\
 &\text{and by PMI we have proven that:}\\
\forall \ n \in \mathbb{N} \ \sum_{j=1}^{n}j&=\frac{n(n+1)}{2}
\end{align*}




\end{flushleft}
\end{document}
