\documentclass{article}
\usepackage[utf8]{inputenc}
\usepackage[english]{babel}
\usepackage{graphicx}
\usepackage{enumerate}
\usepackage{float}
\graphicspath{ {} }
\usepackage{mathtools}
\usepackage{amsmath, amsthm, amssymb, amsfonts}
\usepackage{caption}
\usepackage{fancyhdr}
\pagestyle{fancy}
\fancyhf{}
\rhead{Ty Darnell}
\lhead{Homework 4}

% For derivatives
\newcommand{\deriv}[1]{\frac{\mathrm{d}}{\mathrm{d}x} (#1)}

% For partial derivatives
\newcommand{\pderiv}[2]{\frac{\partial}{\partial #1} (#2)}

% Integral dx
\newcommand{\dx}{\mathrm{d}x}
\begin{document}
\begin{flushleft}
\chead{Problems 1-3}
\section*{Problem 1}
In order for the sum to be greater than or equal to $k$, we need at least $k$ ones. The number of ways to get $k$ ones is:
\[\left(\sum \limits_{i=1}^{n}x_i=k\right) \to {n \choose k}
\]
Then we have to take in account all of the ways to get $k+1,k+2,...,n$ ones.\\
Summing all of these possibilities gives us:
\[\sum \limits_{i=k}^{n}{n \choose i}
\]
\section*{Problem 2}
\begin{align*}
\text{We want to show } \sum \limits_{i=0}^{n}(-1)^i{n \choose i}&=0 \text{ for } n>0 \\
\text{The binomial}& \text{ theorem states:}\\
(x+y)^n&=\sum \limits_{k=0}^{n}{n \choose k}x^k y^{n-k} \quad \forall \ n\in \mathbb{N}  \label{1}\tag{1}\\
\text{Let }& x=-1, y=1, k=i\\
\text{Then } \eqref{1} \text{ becomes }& \sum \limits_{i=0}^{n}{n \choose i}(-1)^i(1)^{n-i}=(-1+1)^n\\
\text{Then we have } \sum \limits_{i=0}^{n}{n \choose i}(-1)^i(1)&=0^n\\
=\sum \limits_{i=0}^{n}{n \choose i}(-1)^i&=0\\
\text{Therefore }
\sum \limits_{i=0}^{n}{n \choose i}(-1)^{i}&=0 \text{ for } n>0
\end{align*}
\pagebreak
\section*{Problem 3}
\begin{enumerate}[(a)]
\item $\dfrac{4{13\choose 5}}{{52\choose 5}} \approx .001980$ \\
\item $\dfrac{13{4\choose 2}{12 \choose 3}4^3}{{52\choose 5}} \approx .422569$\\
\item $\dfrac{{13 \choose 2}{4\choose 2}{4\choose 2}11*4}{{52\choose 5}}\approx .047539$\\
\item $\dfrac{13{4\choose 3}{12 \choose 2}*4^2}{{52\choose 5}}\approx .021128$\\
\item $\dfrac{13*12*4}{{52\choose 5}}\approx .000240$
\end{enumerate}
\pagebreak
\section*{Problem 4}
\chead{Problems 4-6}
\[\dfrac{16*4}{52*51}+\dfrac{4*16}{52*51} = \dfrac{16*4*2}{52*51} \approx .048265
\]
\section*{Problem 5}
Let $E_i$ be the event that the $ith$ couple sit next to each other\\
$P\left(\bigcup_{i=1}^{4}E_i\right)=$P(at least one couple sits together)\\
P(no couples sit together) = 1-P(at least 1 couple sits together)\\
P(no couples sit together) =$1-P\left(\bigcup_{i=1}^{4}E_i\right)$\\
Using the inclusion-exclusion principle we have:
\begin{align*}
P\left(\bigcup_{i=1}^{4}E_i\right)= \sum E_i -\sum E_i \cap E_j +\sum E_i \cap E_j \cap E_k& -\sum E_i \cap E_j \cap E_k \cap E_l\\
P(E_i)=\dfrac{2*7!}{8!}&\\
P(E_i\cap E_j)=\dfrac{2^2*6!}{8!}&\\
P(E_i\cap E_j \cap E_k)=\dfrac{2^3*5!}{8!}&\\
P(E_i\cap E_j \cap E_k \cap E_l)=\dfrac{2^4*4!}{8!}&\\
P\left(\bigcup_{i=1}^{4}E_i\right)={4\choose 1}\dfrac{2*7!}{8!}-{4\choose 2}\dfrac{2^2*6!}{8!}+{4\choose 3}\dfrac{2^3*5!}{8!}-&{4 \choose 4}\dfrac{2^4*4!}{8!}\\
P\left(\bigcup_{i=1}^{4}E_i\right)=1-\frac{3}{7}+\frac{4}{42}-\frac{1}{105}&\\
1-P\left(\bigcup_{i=1}^{4}E_i\right)=1-\left(1-\frac{3}{7}+\frac{4}{42}-\frac{1}{105}\right)&\\
=\frac{3}{7}-\frac{4}{42}+\frac{1}{105}\\
\text{P(no couples sit together)}=\frac{12}{35} =.3428571&
\end{align*}
\section*{Problem 6}
There are $N^n$ equally likely arrangements. There are $\binom{n}{m}$ ways to select $m$ balls for the first compartment. This leaves $N-1$ compartments and $n-m$ balls giving us $(N-1)^{n-m}$ possible arrangements for the remaining balls. Putting this all together we have:
\[\dfrac{\binom{n}{m}(N-1)^{n-m}}{N^n}
\]
\newpage
\section*{Problem 7}
\chead{Problems 7-9}
\[\dfrac{{6\choose 3}{6\choose 3}}{{12\choose 6}}=.4329004
\]
\section*{Problem 8}
There are $(N-r)$ empty parking spaces that could be on adjacent to one side of his car over $(N-1)$ total spaces (since we don't count his car). This then leaves $(N-r-1)$ empty parking spaces that could be adjacent to the other side of his car over $(N-2)$ total spaces. Putting this together we have:
\[ \dfrac{(N-r)(N-r-1)}{(N-1)(N-2)}
\]
\section*{Problem 9}
In order to find the conditional probability of the coin having a heads and tails side given that one side turned up heads, we will use Bayes Rule combined with the Law of Total Probability.
\begin{align*}
\text{3 coins: } &\left\{HH,TT,HT\right\}\\
P(HT|H)&=\dfrac{P(H|HT)P(HT)}{P(H|HT)P(HT)+P(H|{HT}^c)P({HT}^c)}\\
P(HT|H)&=\dfrac{1/2*1/3}{1/2*1/3+1/2*2/3}\\
P(HT|H)&=1/3
\end{align*}
\end{flushleft}
\end{document}
