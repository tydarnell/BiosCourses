\documentclass{article}
\usepackage[utf8]{inputenc}
\usepackage[english]{babel}
\usepackage [autostyle, english = american]{csquotes}
\MakeOuterQuote{"}
\usepackage{graphicx}
\usepackage{enumerate}
\usepackage{float}
\graphicspath{ {} }
\usepackage{mathtools}
\usepackage{amsmath, amsthm, amssymb, amsfonts}
\usepackage{caption}
\usepackage{bm}
\usepackage{fancyhdr}
\pagestyle{fancy}
\fancyhf{}
\rhead{Ty Darnell}
\lhead{667 Homework 7}

% For derivatives
\newcommand{\deriv}[1]{\frac{\mathrm{d}}{\mathrm{d}x} (#1)}

% For partial derivatives
\newcommand{\pderiv}[2]{\frac{\partial #1}{\partial #2}}

% Integral dx
\newcommand{\dx}{\mathrm{d}x}
\newcommand{\cd}{\overset{d}{\to}}
\newcommand{\cp}{\overset{p}{\to}}
\newcommand{\B}{\beta}
\newcommand{\e}{\epsilon}
\newcommand{\limn}{\lim_{n\to \infty}}
\newcommand{\lm}{\lambda}
\newcommand{\sg}{\sigma}
\newcommand{\hb}{\hat{\beta}}
\newcommand{\sumn}{\sum_{i=1}^{n}}
\newcommand{\hth}{\hat{\theta}}
\newcommand{\lra}{\Leftrightarrow}
\newcommand{\prodn}{\prod_{i=1}^{n}}
\newcommand{\dll}[1]{\dfrac{\partial\ell}{\partial{#1}}}
\newcommand{\mle}{\hat{\theta}_{MLE}}
\newcommand{\mm}{\hat{\theta}_{MM}}
\newcommand{\sumx}{\sum_{i=1}^{n}x_i}
\newcommand{\ta}{\theta}
\newcommand{\qe}{ \ ?\ }
\newcommand{\dt}{\pderiv{}{\ta}}
\newcommand{\lt}[1]{\log(f(#1|\ta))}
\newcommand{\lx}{\lambda(x)}
\newcommand{\samp}{X_1,\dots,X_n \sim}
\newcommand{\te}{\theta_1}
\newcommand{\xm}{x_{(1)}}
\newcommand{\sn}{(\sg^2)}
\newcommand{\pow}{\B(\ta)}
\newcommand{\hyp}[2]{H_0: #1 \text{ vs } H_1: #2}
\newcommand{\pois}[2]{\dfrac{e^{-#1}{#1}^{#2}}{{#2}!}}
\newcommand{\mlr}{\dfrac{f(x|\ta_2)}{f(x|\ta_1)}}
\newcommand{\al}{\alpha}
\newcommand{\bx}{\bar{x}}
\allowdisplaybreaks
\begin{document}
\begin{flushleft}
\section*{Problem 13.1}
\subsection*{Part 0}
\begin{multline*}\\
\textbf{Simple Statistics}\\
\text{Clinic 1, Active Treatment}\\
\begin{array}{llll}
\hline
Visit & N & Mean & SD\\
\hline
0 & 27 & 0.33 & 0.48 \\ 
1 & 27 & 0.52 & 0.51 \\ 
2 & 27 & 0.59 & 0.50 \\ 
3 & 27 & 0.63 & 0.49 \\ 
4 & 27 & 0.44 & 0.51 \\ 
\hline
\end{array}\\
\text{Clinic 1, Placebo}\\
\begin{array}{llll}
\hline
Visit & N & Mean & SD\\
\hline
0 & 29 & 0.31 & 0.47 \\ 
1 & 29 & 0.41 & 0.50 \\ 
2 & 29 & 0.34 & 0.48 \\ 
3 & 29 & 0.41 & 0.50 \\ 
4 & 29 & 0.31 & 0.47 \\ 
\hline
\end{array}\\
\text{Clinic 2, Active Treatment}\\
\begin{array}{llll}
\hline
Visit & N & Mean & SD\\
\hline
0 & 27 & 0.56 & 0.51 \\ 
1 & 27 & 0.85 & 0.36 \\ 
2 & 27 & 0.81 & 0.40 \\ 
3 & 27 & 0.81 & 0.40 \\ 
4 & 27 & 0.78 & 0.42 \\ 
\hline
\end{array}\\
\text{Clinic 2, Placebo}\\
\begin{array}{llll}
\hline
Visit & N & Mean & SD\\
\hline
0 & 28 & 0.61 & 0.50 \\ 
1 & 28 & 0.57 & 0.50 \\ 
2 & 28 & 0.43 & 0.50 \\ 
3 & 28 & 0.50 & 0.51 \\ 
4 & 28 & 0.57 & 0.50 \\ 
\hline
\end{array}\\
\textbf{Covariance Matrices}\\
\text{Clinic 1, Active Treatment}\\
\left[\begin{array}{rrrrr}
0.23 & 0.17 & 0.10 & 0.13 & 0.08 \\ 
0.17 & 0.26 & 0.10 & 0.12 & 0.07 \\ 
0.10 & 0.10 & 0.25 & 0.11 & 0.15 \\ 
0.13 & 0.12 & 0.11 & 0.24 & 0.09 \\ 
0.08 & 0.07 & 0.15 & 0.09 & 0.26 \\  
\end{array}\right]\\\\
\text{Clinic 1, Placebo}\\
\left[\begin{array}{rrrrr}
0.23 & 0.17 & 0.10 & 0.13 & 0.08 \\ 
0.17 & 0.26 & 0.10 & 0.12 & 0.07 \\ 
0.10 & 0.10 & 0.25 & 0.11 & 0.15 \\ 
0.13 & 0.12 & 0.11 & 0.24 & 0.09 \\ 
0.08 & 0.07 & 0.15 & 0.09 & 0.26 \\ 
\end{array}\right]\\
\text{Clinic 2, Active Treatment}\\
\left[\begin{array}{rrrrr}
0.26 & 0.09 & 0.03 & 0.07 & 0.05 \\ 
0.09 & 0.13 & 0.05 & 0.05 & 0.08 \\ 
0.03 & 0.05 & 0.16 & 0.12 & 0.07 \\ 
0.07 & 0.05 & 0.12 & 0.16 & 0.07 \\ 
0.05 & 0.08 & 0.07 & 0.07 & 0.18 \\  
\end{array}\right]\\
\text{Clinic 2, Placebo}\\
\left[\begin{array}{rrrrr}
0.26 & 0.09 & 0.03 & 0.07 & 0.05 \\ 
0.09 & 0.13 & 0.05 & 0.05 & 0.08 \\ 
0.03 & 0.05 & 0.16 & 0.12 & 0.07 \\ 
0.07 & 0.05 & 0.12 & 0.16 & 0.07 \\ 
0.05 & 0.08 & 0.07 & 0.07 & 0.18 \\ 
\end{array}\right]\\
\end{multline*}
\subsection*{Part 1}
Model for the log odds that respiratory status is classified as good\\
Assuming separate pairwise log odds ratios among the five binary responses\\
where $i=0$ is the placebo (reference) and $i=1$ is the active treatment\\
$j=0,1,2,3,4$ for the $j_{th}$ occasion time 0 (baseline) is the reference \\
Under the constraint that the treatment groups have the same mean at time 0 we have $\beta_0=\gamma_{0j}=\gamma_{i0}=0$\\
$\eta_{ij}=\mu+\beta_jI(\text{time}_j)+\gamma_{ij}I(\text{time}_j)*I(\text{treatment}_i)$\\
\begin{multline*}\\
\begin{array}{ccc}
\hline
Parameter & estimate & \hat{se}\\
\hline
\hat{\mu}& -.199 & .191\\
\hat{\beta_1}& .151 &.262\\
\hat{\beta_2}& -.274 &.282\\
\hat{\beta_3}&.012&.263 \\
\hat{\beta_4}& -.058&.257 \\
\hat{\gamma_{11}}&.838&.337\\
\hat{\gamma_{12}}&1.347& .380\\
\hat{\gamma_{13}}&1.154&.361\\
\hat{\gamma_{14}}&.718&.355\\
\hline
\end{array}\\
\text{Treatment effect}=\boldsymbol{\delta}=(\gamma_{11},\gamma_{12},\gamma_{13},\gamma_{14})^T\\
\boldsymbol{\delta} \text{ represents the treament effect on changes in the log odds that respiratory status is classified as good}\\
\text{Testing the null hypothesis
of no effect of treatment on changes in the log odds that respiratory status}\\ 
\text{is classified as good based on the empirical standard errors}\\
\text{Wald } \chi^2 \text{ Test}\\
H_0: \boldsymbol{\delta}=0\\
H_1: \boldsymbol{\delta}\neq 0\\
\chi^2=16.91 \text{ with } df=4\\
\text{p-value}=.002<.05 \text{Thus reject } H_0\\
\end{multline*}
\subsection*{Part 2}
Since the null hypothesis was rejected that there is evidence to suggest a statistically significant effect of treatment on changes in the log odds that respiratory status is classified as good
\subsection*{Part 3a}
Model for the log odds that respiratory status is classified as good\\
$\eta_{ijk}=\mu+\beta_j+\alpha_k+\gamma_{ij}+\tau_{jk}+\lambda_{ijk}$\\
where $i=0$ is the placebo (reference) and $i=1$ is the active treatment\\
$j=0,1,2,3,4$ for the $j_{th}$ occasion time 0 (baseline) is the reference \\
$k=1,2$ is the $k_{th}$ center with center 1 as the reference\\
\begin{multline*}\\
\text{Treatment effect}=\boldsymbol{\delta}=(\lambda_{112},\lambda_{122},\lambda_{132},\lambda_{142})^T\\
\boldsymbol{\delta} \text{ represents the treament effect on changes in the log odds that respiratory status is classified as good}\\
\begin{array}{ccc}
\hline
Parameter & estimate & \hat{se}\\
\hline
\hat{\lambda_{112}}&1.203&.759\\
\hat{\lambda_{122}}&.862&.803\\
\hat{\lambda_{132}}&.749&.761\\
\hat{\lambda_{142}}&.507&.773\\
\hline
\end{array}\\
\text{Testing if the effect of treatment is the same in the two clincs}\\
\text{Wald } \chi^2 \text{ Test}\\
H_0: \boldsymbol{\delta}=0 \text{ (The treament effect is the samin in the two clinics)}\\
H_1: \boldsymbol{\delta}\neq 0\\
\chi^2=3.10 \text{ with } df=4\\
\text{p-value}=.541>.05 \text{ Thus fail to reject } H_0\\
\end{multline*}
Since we failed to reject the null hypothesis there is not enough evidence to suggest a statistically significant difference in  effect of treatment between the two clinics on changes in the log odds that respiratory status is classified as good
\subsection*{Part 3b}
We drop the 3 variable interactions since the results from part a showed it was non-significant.\\
Model for the log odds that respiratory status is classified as good\\
$\eta_{ijk}=\mu+\beta_j+\alpha_k+\gamma_{ij}+\tau_{jk}$\\
where $i=0$ is the placebo (reference) and $i=1$ is the active treatment\\
$j=0,1,2,3,4$ for the $j_{th}$ occasion time 0 (baseline) is the reference \\
$k=1,2$ is the $k_{th}$ center with center 1 as the reference\\
\begin{multline*}\\
\begin{array}{ccc}
\hline
Parameter & estimate & \hat{se}\\
\hline
\hat{\mu}& -.771&.289\\
\hat{\beta_1}&.202&.295\\
\hat{\beta_2}& -.046&.330\\
\hat{\beta_3}&.289&.297\\
\hat{\beta_4}&-.127&.356\\
\hat{\alpha_2}&1.103&.400\\
\hat{\gamma_{11}}&.892&.348\\
\hat{\gamma_{12}}&1.389&.389\\
\hat{\gamma_{13}}&1.195&.368\\
\hat{\gamma_{14}}&.774&.380\\
\hat{\tau_{12}}&-.071&.417\\
\hat{\tau_{22}}&-.437&.470\\
\hat{\tau_{32}}&-.531&.428\\
\hat{\tau_{42}}&.153&.461\\
\hline
\end{array}\\
\end{multline*}
$\hat{\mu}$ is the log odds ratio of good respiratory status at baseline for the placebo group at center 1.\\
$\hat{\beta_j}$ the difference in the log odds ratio of good respiratory status at occasion j relative to baseline for patients receiving placebo at center 1.\\
$\hat{\alpha_2}$ is the difference in the log odds ratio of good respiratory status for patients at center 2 compared to center 1\\
$\hat{\gamma_{1j}}$ is the interaction effect of treatment and time on the log odds ratio of good respiratory status.\\
$\hat{\tau_{j2}}$ is the interaction effect of time and center 2 on the log odds ratio of good respiratory status.\\
\subsection*{Part 4}
Table of estimated probabilities that respiratory status is classified as good as a function of both time and treatment group clinic
\begin{multline*}\\
\begin{array}{ccccc}
\hline
 Time &  \text{Clinic 1} & & \text{Clinic 2}\\
  & Placebo & Active & Placebo &Active\\
 \hline
 0 &.316 &.316& .582 &.582\\
 1 & .361&.580&.614&.795\\
 2&.306&.639&.614&.775\\
 3&.382&.671&.522&.783\\
 4&.289&.589&.469&.756\\
 \hline
\end{array}\\
\end{multline*}
Based on the table it appears that the treatment has an effect on the probability good respiratory status within in each clinic, but it does not appear that clinic has a significant effect on the probability good respiratory status
\end{flushleft}
\end{document}
