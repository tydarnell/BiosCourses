\documentclass{article}
\usepackage[utf8]{inputenc}
\usepackage[english]{babel}
\usepackage [autostyle, english = american]{csquotes}
\MakeOuterQuote{"}
\usepackage{graphicx}
\usepackage{enumerate}
\usepackage{float}
\graphicspath{ {} }
\usepackage{mathtools}
\usepackage{amsmath, amsthm, amssymb, amsfonts}
\usepackage{caption}
\usepackage{bm}
\usepackage{fancyhdr}
\pagestyle{fancy}
\fancyhf{}
\rhead{Ty Darnell}
\lhead{Bios 667 Homework 1}
% For derivatives
\newcommand{\deriv}[1]{\frac{\mathrm{d}}{\mathrm{d}x} (#1)}

% For partial derivatives
\newcommand{\pderiv}[2]{\frac{\partial #1}{\partial #2}}

% Integral dx
\newcommand{\dx}{\mathrm{d}x}
\newcommand{\cd}{\overset{d}{\to}}
\newcommand{\cp}{\overset{p}{\to}}
\newcommand{\B}{\beta}
\newcommand{\e}{\epsilon}
\newcommand{\limn}{\lim_{n\to \infty}}
\newcommand{\lm}{\lambda}
\newcommand{\sg}{\sigma}
\newcommand{\hb}{\hat{\beta}}
\newcommand{\sumn}{\sum_{i=1}^{n}}
\newcommand{\hth}{\hat{\theta}}
\newcommand{\lra}{\Leftrightarrow}
\newcommand{\prodn}{\prod_{i=1}^{n}}
\newcommand{\dll}[1]{\dfrac{\partial\ell}{\partial{#1}}}
\newcommand{\mle}{\hat{\theta}_{MLE}}
\newcommand{\mm}{\hat{\theta}_{MM}}
\newcommand{\sumx}{\sum_{i=1}^{n}x_i}
\newcommand{\ta}{\theta}
\newcommand{\qe}{ \ ?\ }
\newcommand{\dt}{\pderiv{}{\ta}}
\newcommand{\lt}[1]{\log(f(#1|\ta))}
\newcommand{\lx}{\lambda(x)}
\newcommand{\samp}{X_1,\dots,X_n \sim}
\newcommand{\te}{\theta_1}
\newcommand{\xm}{x_{(1)}}
\newcommand{\sn}{(\sg^2)}
\newcommand{\pow}{\B(\ta)}
\newcommand{\hyp}[2]{H_0: #1 \text{ vs } H_1: #2}
\newcommand{\pois}[2]{\dfrac{e^{-#1}{#1}^{#2}}{{#2}!}}
\newcommand{\mlr}{\dfrac{f(x|\ta_2)}{f(x|\ta_1)}}
\newcommand{\al}{\alpha}
\newcommand{\bx}{\bar{x}}
\newcommand{\by}{\bar{y}}
\newcommand{\hu}{\hat{\mu}}
\allowdisplaybreaks
\begin{document}
\begin{flushleft}

\section*{Problem 1}
\begin{enumerate}[(a)]
\item
\begin{tabular}{l l l l l}
\hline
$i$ & group & n & Mean & SD\\
\hline
1 & Non-smoker & 21 & 3.78 & 1.79\\
2 & Light smoker & 21 & 3.23 & 1.86\\
3 & Heavy smoker & 21 & 2.59 &1.82\\
\hline
\end{tabular}\\
$Y_{ij}$ denotes the lung function measure for the $j_{th}$ subject in the $i_{th}$ smoking group\\

$K=3$ total number of smoking groups\\

$N=63$ total number of subjects\\

$H_0: \mu_1=\mu_2=\mu_3$ The lung function measure populations means are equal across the smoking groups.\\

$H_1:$ The lung function measure population means are not equal across the smoking groups.\\

$s^2_p = \dfrac{\sum_{i=1}^{3}(n_i-1)s_i^2}{\sum_{i=1}^{3}(n_i-1)}\approx 3.33$\\

$\bar{Y} = \dfrac{\sum_{i=1}^{3}\sum_{j=1}^{n_i} Y_{ij} }{N}=3.2$\\

Under $H_0: \hat{\sigma}^2=\dfrac{\sum_{i=1}^{3}n_i(\bar{Y}_i-\bar{Y})^2}{K-1}\approx 7.45$\\

Under $H_0: F \equiv \dfrac{\hat{\sigma}^2}{s^2_p}\sim F_{K-1,N-K}$\\

$F$ test statistic $=\dfrac{\hat{\sigma}^2}{s^2_p}\approx 2.24$\\

Critical Region $C_{\alpha}=\{F:F>F_{2,60;1-\alpha}\}$\\

Using $\alpha=.05$\\

Quantile ${F^{-1}}_{2,60;.95} \approx 3.15$\\

$2.24<3.14$ The test statistic is outiside of the critical region thus fail to reject $H_0$\\

p-value $= 1-F_{2.24,2,60}\approx .12>.05$ The p-value is larger than $\alpha$\\

Not enough evidence to reject the null hypothesis that the population means for lung function measure are equal across all smoking groups.\\

\begin{tabular}{l l l l l l}
	\textbf{Anova Table}\\
	\hline
	Source of Variation & SS & df & Mean Square & F & P-value \\ \hline
	Among groups & 14.9 &  2 & $\hat{\sigma}^2=7.45$ & 2.24 & .12 \\
	Within Groups & 199.8& 60 & $s^2_p=3.33$ \\ \hline
	Total & 214.7 & $62$ \\ \hline
\end{tabular}

\pagebreak
\item

Constraint on model:  $\mu_i=\alpha_1+\alpha_2i$\\

Using $\bar{y}_i$ to estimate $\mu_i$\\

$\bar{y}_1=\alpha_1+\alpha_2*1$\\

$\bar{y}_2=\alpha_1+\alpha_2*2$\\

$\bar{y}_3=\alpha_1+\alpha_2*3$\\

Using collapsed model to obtain $\hat{\alpha}_1$ and $\hat{\alpha}_2$\\
$\bar{y}_i= [3.78,3.23,2.59]^T$  $i=[1,2,3]$\\
Essence Matrix  $\left[\begin{array}{rr}
1 & 1 \\ 
1 & 2 \\ 
1 & 3 \\ 
\end{array}
\right]$\\
Grand mean= $\by=3.2 \quad \bx=2$\\

$SXX=\sum(x_i-\bx)^2=2$\\ 
$SXY=\sum(x_i-\bx)y_i=-1.19$\\
$SYY=\sum(y_i-\by)^2=.7094$\\
$\hat{\alpha}_2=\frac{SXY}{SXX}=\frac{-1.19}{2}=-.595$\\

$\hat{\alpha}_1=\by-\hat{\alpha}_2 \bx=3.2-2*-.595=4.39$\\

$\hat{\alpha}_1 = 4.39 \quad \hat{\alpha}_2 = -.595$\\
$\tilde{RSS}=SYY-\hat{\alpha}_2^2 SXX=.00135$\\

Actual Model: $y_{ij}=\alpha_1+\alpha_2*i+\epsilon_{ij}$\\

In order to obtain the SE for the estimates of the actual model we will need to compute SSE of the actual model using $\tilde{RSS}$ from the collapsed model and $SS_{Within}$ from the anova table from part a.\\

$SSE=SS_{Within}+63*\tilde{RSS}$\\

$SSE=199.8+63*.00135= 199.8851$\\

Then use SSE to compute $\hat{\sigma}^2$\\

$\hat{\sigma}^2=SSE/(N-2)\approx 3.28$\\


Next create an X matrix and calculate $(X^{'}X)^{-1}$\\

$X_{63\times 2}=\left[
\begin{array}{rr}
1 & 1 \\ 
\vdots & \vdots \\ 
1 & 2 \\ 
\vdots & \vdots \\ 
1 & 3 \\ 
\vdots & \vdots \\ 
\end{array}
\right]$\\

$(X^{'}X)^{-1}=\left[
\begin{array}{rr}
0.1111 & -0.0476 \\ 
-0.0476 & 0.0238 \\ 
\end{array}
\right]$\\


Use the diagonal of $(X^{'}X)^{-1}$ matrix to obtain the standard error of the estimates\\

Standard error of estimates=$\sqrt{\hat{\sigma}^2*(X^{'}X)^{-1}_{ii}}=[.60,.28]$\\

$\hat{\sigma}^2\approx 3.28$\\

\begin{tabular}{l l l}
	\hline
	& Estimate & Std Error\\
	\hline
	$\hat{\alpha_1}$ & 4.39 & .60 \\
	$\hat{\alpha_2}$ & -.60 & .28 \\
	\hline
\end{tabular}

\item
$\mu_i=\beta_1+\beta_2i+\beta_3i^2$\\

$H_0:\beta_2=\beta_3=0$\\

since $\beta_2=\beta_3=0$ we have $\mu_i=\beta_1+ 0$\\

Equivalent to testing $\mu_1=\mu_2=\mu_3 (=\beta_1)$\\

This is the same as the test from part a\\

Thus under $H_0: F \approx 2.24\sim F_{2,60}$\\

Using $\alpha=.05$\\

Critical Region $C_{\alpha}=\{F:F>F_{2,60;.95}\}$\\

Quantile ${F^{-1}}_{2,60;.95} \approx 3.15$\\

$2.24<3.14$ The test statistic is outiside of the critical region thus fail to reject $H_0$\\

p-value $= 1-F_{2.24,2,60}\approx .12>.05$ The p-value is larger than $\alpha$\\

Not enough evidence to reject the null hypothesis that the population means for lung function measure are equal across all smoking groups.
\end{enumerate}


\section*{Problem 2}
	
\begin{enumerate}[(a)]
	
	\item 
\begin{multline*}\\
\bx =2 \quad SXX=\sum(x_i-\bx)^2=6\\
SXY=\sum(x_i-\bx)y_i=-y_1-y_2+2y_4\\
\hb_2=\dfrac{SXY}{SXX}=(1/6)(-y_1-y_2+2y_4)\\
\hb_1=\by-\hb_2 \bx\\
\hb_1=(1/4)(y_1+y_2+y_3+y_4)-2*(1/6)(-y_1-y_2+2y_4)\\
\hb_1=(1/4+1/3)y_1+(1/4+1/3)y_2+(1/4)y_3+(1/4-2/3)y_4\\
=(7/12)y_1+(7/12)y_2+(1/4)y_3-(5/12)y_4\\
\hb_1=.583y_1+.583y_2+.25y_3-.417y_4\\
.583+.583+.25-.417=1\\
\end{multline*}

	\item 
\begin{multline*}\\
E(\hb_1)=(7/12)E(y_1)+(7/12)E(y_2)+(1/4)E(y_3)-(5/12)E(y_4)\\
=(7/12)(1)+(7/12)(1)+(1/4)(2)-(5/12)(8)\\
=20/12-40/12=-20/12\approx -1.667\\
E(\hb_1)\approx -1.667\\
\end{multline*}


\item 
\begin{multline*}\\
Var(\hb_1)=Var[(7/12)y_1+(7/12)y_2+(1/4)y_3-(5/12)y_4]\\
\text{Since } Cov(Y_i,Y_j)=0 \text{ we have:}\\
Var(\hb_1)=(49/144)Var(y_1)+(49/144)Var(y_2)+(1/16)Var(y_3)+(25/144)Var(y_4)\\
=(49/144)1+(49/144)2+(1/16)4+(25/144)5=\frac{77}{36}\approx 2.139\\
Var(\hb_1)\approx 2.139\\
\end{multline*}

\item 
\begin{multline*}\\
\hb_2=\dfrac{SXY}{SXX}=(1/6)(-y_1-y_2+2y_4)\\
=(-1/6)y_1+(-1/6)y_2+(1/3)y_4\\
\hb_2=-.167y_1-.167y_2+.333y_4\\
E(\hb_2)=E[(-1/6)y_1+(-1/6)y_2+(1/3)y_4]\\
=(-1/6)*1+(-1/6)*1+(1/3)*8\\
E(\hb_2)=2.333\\
Var(\hb_2)=Var[(-1/6)y_1+(-1/6)y_2+(1/3)y_4]\\
\text{Since } Cov(Y_i,Y_j)=0 \text{ we have:}\\
Var(\hb_2)=(1/36)Var(y_1)+(1/36)Var(y_2)+(1/9)Var(y_2)\\
=(1/36)*1+(1/36)*2+(1/9)*5=\frac{23}{36}\approx .639\\
Var(\hb_2)\approx .639\\
\end{multline*}

\item 
\begin{multline*}\\
Var(\hb_1+\hb_2)=Var\left(\frac{1}{12}[7y_1+7y_2+3y_3-5y_4-2y_1-2y_2+4y_4]\right)\\
=\frac{1}{144}Var(5y_1+5y_2+3y_3-y_4)\\
\text{Since } Cov(Y_i,Y_j)=0 \text{ we have:}\\
Var(\hb_1+\hb_2)=\frac{1}{144}(25Var(y_1)+25Var(y_2)+9Var(y_3)+Var(y_4))\\
Var(\hb_1+\hb_2)=\frac{1}{144}(5*1+5*2+3*4-1*5)=\frac{11}{72}\\
Cov(\hb_1,\hb_2)=(1/2)[Var(\hb_1)+Var(\hb_2)-Var(\hb_1+\hb_2)]\\
Cov(\hb_1,\hb_2)=\frac{1}{2}\left(\frac{77}{36}+\frac{23}{36}-\frac{11}{72}\right)=\frac{189}{144}=1.3125\\
\end{multline*}

\item 
\begin{multline*}\\
\hu_i=\hb_1+\hb_2x_i\\
\hu_3=\hb_1+\hb_2x_3=\hb_1+2\hb_2\\
\hu_3=(7/12)y_1+(7/12)y_2+(1/4)y_3-(5/12)y_4+2[(-1/6)y_1+(-1/6)y_2+(1/3)y_4]\\
=\frac{7-4}{12}y_1+\frac{7-4}{12}y_2+\frac{1}{4}y_3+\frac{8-5}{12}y_4=\frac{1}{4}y_1+\frac{1}{4}y_2+\frac{1}{4}y_3+\frac{1}{4}y_4\\
\hu_3=\frac{1}{4}(y_1+y_2+y_3+y_4)\\
E(\hu_3)=(1/4)E\left(y_1+y_2+y_3+y_4\right)\\
=(1/4)[E(y_1)+E(y_2)+E(y_3)+E(y_4)]=\frac{1+1+2+8}{4}=3\\
E(\hu_3)=3\\
\end{multline*}

\item 
\begin{multline*}\\
R_i=Y_i-\hu_i\\
R_3=y_3-\hu_3=y_3-\frac{1}{4}(y_1+y_2+y_3+y_4)\\
R_3=\frac{-1}{4}(y_1+y_2-3y_3+y_4)\\
E(R_3)=E(y_3)-E(\hu_3)=2-3=-1\\
E(R_3)=-1\\
Var(R_3)=Var\left(\frac{-1}{4}(y_1+y_2-3y_3+y_4)\right)\\
\text{Since } Cov(Y_i,Y_j)=0 \text{ we have:}\\
Var(R_3)=\frac{1}{16}[Var(y_1)+y_2+9Var(y_3)+Var(y_4)]=\frac{1+2+9*4+5}{16}=\frac{11}{4}\\
Var(R_3)=2.75\\
E(R^2_3)=Var(R_3)+E(R_3)^2=2.75+3^2=11.75\\
\end{multline*}


\end{enumerate}


\end{flushleft}
\end{document}
